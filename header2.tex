% ============ Index ============
\usepackage{imakeidx}
\makeindex[intoc=true, columns=2, columnseprule=true, options=-s latex/indexstyles.ist]

% ============ Judul Bab/Bagian ============
\usepackage{titlesec}

% Judul BAB (chapter)
\titleformat{\chapter}[display]
  {\normalfont\huge\bfseries\centering}
  {\chaptertitlename\ \thechapter}{0pt}{\Huge}
\titlespacing*{\chapter}{0pt}{0pt}{20pt}

% Judul BAGIAN (part)
\titleformat{\part}[display]
  {\normalfont\Huge\bfseries\centering}
  {\partname\ \thepart}{0pt}{\Huge}
\titlespacing*{\part}{0pt}{0pt}{20pt}

% --- Kontrol page break untuk PART saja ---
% (Biarkan LaTeX menangani page break untuk chapter secara default)
\usepackage{etoolbox}
\makeatletter
\AtBeginDocument{%
  % Nonaktifkan page break hanya untuk PART
  \pretocmd{\part}{\begingroup\let\cleardoublepage\relax \let\clearpage\relax}{}{}
  \patchcmd{\@endpart}{\vfil\newpage}{}{}{}
  \apptocmd{\@endpart}{\endgroup}{}{}
}
\makeatother

% ============ Callout (kotak catatan) ============
\usepackage[most]{tcolorbox}
\usepackage{xcolor}
\definecolor{MyCalloutHeader}{HTML}{58771F}
\definecolor{MyCalloutBody}{HTML}{D6DEBF}
\definecolor{MyCalloutTitle}{HTML}{FFFFFF}

\tcbset{
  quarto-callout-note/.style={
    colback=MyCalloutBody,
    colframe=MyCalloutHeader,
    colbacktitle=MyCalloutHeader,
    coltitle=MyCalloutTitle,
    fonttitle=\itshape
  }
}