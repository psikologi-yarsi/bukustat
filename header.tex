% ============ Index ============
\usepackage{imakeidx}
\makeindex[intoc=true, columns=2, columnseprule=true, options=-s latex/indexstyles.ist]

% ============ Pengaturan Font (opsional; sudah ada di YAML) ============
% \usepackage{fontspec}
% \setmainfont{Ubuntu Light}
% \setsansfont{Ubuntu Light}
% \setmonofont{Ubuntu Mono}

% ============ Judul Bab/Bagian ============
\usepackage{titlesec}

% Judul BAB (chapter)
\titleformat{\chapter}[display]
  {\normalfont\huge\bfseries\centering}
  {\chaptertitlename\ \thechapter}{0pt}{\Huge}
\titlespacing*{\chapter}{0pt}{0pt}{20pt}

% Judul BAGIAN (part) — pusat vertikal di HALAMAN SENDIRI
\titleformat{\part}[display]
  {\normalfont\Huge\bfseries\centering}
  {\partname\ \thepart}{0pt}{\Huge}
% before = \fill, after = \fill  -> memusatkan vertikal
\titlespacing*{\part}{0pt}{\fill}{\fill}

% ============ Callout (kotak catatan) ============
\usepackage[most]{tcolorbox}
\usepackage{xcolor}
\definecolor{MyCalloutHeader}{HTML}{58771F}
\definecolor{MyCalloutBody}{HTML}{D6DEBF}
\definecolor{MyCalloutTitle}{HTML}{FFFFFF}

\tcbset{
  quarto-callout-note/.style={
    colback=MyCalloutBody,
    colframe=MyCalloutHeader,
    colbacktitle=MyCalloutHeader,
    coltitle=MyCalloutTitle,
    fonttitle=\itshape
  }
}

% ============ Kontrol page break yang konsisten ============
\usepackage{etoolbox}
\makeatletter
\AtBeginDocument{%
  % 1) Pastikan PART berakhir dengan vfill + newpage (judul tetap center-middle)
  \renewcommand\@endpart{\vfil\newpage}

  % 2) Pastikan CHAPTER selalu mulai di halaman baru
  \let\orig@chapter\chapter
  \renewcommand{\chapter}{\clearpage\orig@chapter}
}
\makeatother

\usepackage{qrcode}

% kotak angka untuk PDF
\newcommand{\boxnum}[1]{%
  \fbox{\rule{0pt}{1.2em}\rule{1.2em}{0pt}#1}%
}

%\usepackage{fontspec}

% Body menggunakan file lokal
%\setmainfont{Ubuntu-light.ttf}[
%  Path=fonts/,
%  Extension = .ttf,
%  BoldFont  = Ubuntu-Bold.ttf,
%  ItalicFont= Ubuntu-LightItalic.ttf,
%  Ligatures = TeX
%]

% Sans menggunakan varian Light
%\setsansfont{Ubuntu-Light.ttf}[
%  Path=fonts/,
%  Extension = .ttf,
%  BoldFont  = Ubuntu-Bold.ttf,  % opsional: fallback bold
%  Ligatures = TeX
%]

% Mono (opsional)
%\setmonofont{Ubuntu-Light.ttf}[
%  Path=fonts/,
%  Extension = .ttf
%]